\documentclass[11pt, a4paper]{article}

\usepackage[T1]{fontenc}
\usepackage[utf8]{inputenc}
\usepackage[english]{babel}

% --- Fonts --------------------------------------------------------------------
% Serif (main text)
\usepackage{erewhon}

% Sans (headings, small UI, captions). sfdefault -> \sffamily becomes default sans.
\usepackage[sfdefault]{FiraSans}

% Monospace (code/listings)
\usepackage{FiraMono}

% Math font: newtxmath is a dependable pdfLaTeX math package.
% Some useful options you can try: upint, varg, slantedGreek, smbb, cmintegrals.
% Example: \usepackage[upint,slantedGreek]{newtxmath}
\usepackage[upint]{newtxmath}

% Micro-typography (always recommended for professional docs)
\usepackage{microtype}
\microtypesetup{protrusion=true,expansion=true}

% Math, Physics
\usepackage{mathtools}
\usepackage{amsfonts}
\usepackage{physics, tensor}

% Margins and spacing
\usepackage{geometry}
\geometry{left=25mm, right=25mm, top=30mm, bottom=30mm, headheight=15pt}
\usepackage{setspace}
\singlespacing % or \onehalfspacing or \doublespacing

% Misc
\usepackage{xcolor}
\usepackage[none]{hyphenat}
\usepackage[colorlinks=true,
  linkcolor=black,
  citecolor=blue!60!black,
  urlcolor=blue!60!black
]{hyperref}
\usepackage{cleveref}

% Titles
\usepackage{titlesec}
\titleformat
{\section} % command
{\sffamily\bfseries\Large} % format
{\thesection.} % label
{1ex} % sep
{} % before-code
[] % after-code

% Title page
\usepackage{titling}
\pretitle{\LARGE\bfseries\sffamily\noindent}
\posttitle{
    \par
    \noindent\makebox[\linewidth]{\rule{\linewidth}{1pt}}
    \vspace{2ex}
}
\preauthor{
    \par\large\bfseries\sffamily\noindent
}
\postauthor{\par}
\predate{\par\vspace{1ex}}
\postdate{\par\vspace{2ex}}

% Headers and Footers
\usepackage{fancyhdr}
\pagestyle{fancy}
\fancyhead[L]{Title or Section}
\fancyhead[R]{Page \thepage}
\fancyfoot[C]{}

% References
\usepackage[
  backend=biber,
  style=numeric-comp,
  sorting=none,
  giveninits=true,
  doi=false,
  isbn=false
]{biblatex}
\usepackage{csquotes}
\addbibresource{references/figures.bib}
\addbibresource{references/papers.bib}
% 1. If there is an arXiv/eprint, hide URL + urldate
\AtEveryBibitem{%
  \iffieldundef{eprint}
    {}
    {\clearfield{url}\clearfield{urldate}}%
}

% 3. Make labels ([1], [2], …) bold and aligned
\DeclareFieldFormat{labelnumber}{\textbf{#1}}
\setlength{\biblabelsep}{0.75em}

% 4. Slightly smaller, tighter bib font
\renewcommand*{\bibfont}{\small}

% 5. Don’t print “URL:” in front of URLs (when they appear)
\DeclareFieldFormat{url}{\url{#1}}


\newcommand{\ii}{\mathrm{i}}
\newcommand{\Li}{\operatorname{Li}}


% Figures
\usepackage{graphicx, float}
\graphicspath{{figures/}}
\usepackage{caption}
\usepackage{subcaption}

% Tables
\usepackage{booktabs}

\title{Notes on Local \texorpdfstring{$\mathbb{F}_1$}{F1}}
\author{Rishi Raj}
\date{
    \noindent\textit{Sorbonne Université, Paris, FR}\\[0.2cm]
    \href{mailto:rishiraj.1012exp@gmail.com}{\texttt{rishi.raj@sorbonne-universite.fr}}
}

\numberwithin{equation}{section}

\renewenvironment{abstract}
 {\par\noindent\textbf{\abstractname}\ \ignorespaces\\[1ex]}
 {\par\medskip}

\begin{document}

\thispagestyle{empty}
\maketitle

\begin{abstract}
  Lorem ipsum dolor sit amet, consectetur adipiscing elit, sed do eiusmod tempor incididunt ut labore et dolore magna aliqua. Tempor orci eu lobortis elementum nibh tellus. Varius vel pharetra vel turpis. Viverra maecenas accumsan lacus vel facilisis volutpat est. Non curabitur gravida arcu ac tortor. Mattis nunc sed blandit libero volutpat sed cras ornare. Etiam tempor orci eu lobortis. Netus et malesuada fames ac turpis egestas maecenas pharetra. Velit dignissim sodales ut eu sem integer. Non arcu risus quis varius quam quisque id diam vel. Lacus suspendisse faucibus interdum posuere lorem ipsum. Diam phasellus vestibulum lorem sed risus. Etiam sit amet nisl purus. Ut sem nulla pharetra diam. Nibh nisl condimentum id venenatis a.
\end{abstract}

\noindent\today

\tableofcontents

\setlength{\parindent}{1ex}
\setlength{\parskip}{2ex}
\raggedright

\section{Nagell's algorithm on \texorpdfstring{$\mathbb{F}_1$}{F1}}
We start from the equation of the local mirror of $\mathbb{F}_1$ defined as the elliptic curve,
\begin{equation}
  x + y + \frac{1}{x y} + \frac{m}{x} = \frac{1}{u},
\end{equation}
where $m$ and $u$ are a priori free complex parameters. After multiplying by $x y$, we have
\begin{equation}
  f(x, y) = x^2 y + x y^2 - \frac{x y}{u} + m y + 1 = 0.
\end{equation}
We can eliminate the constant term by shifting $y \to y - \frac{1}{m}$,
\begin{equation}
  f(x, y) = x^2 y + x y^2 - \frac{x^2}{m} - \frac{(2u + m) x y}{u m} + m y + \frac{(m + u)x}{m^2 u}.
\end{equation}
We perform a series of affine transformations on $x$ and $y$ known as Nagell's algorithm as described in \cite{ConnellHandbook} to bring the curve into Weierstrass form. Note that in order for the curve to be elliptic, $m$ and $u$ cannot vanish simultaneously. Since we would be interested in the $m \to 0$ limit (Local $\mathbb{P}^2$), we switch $x \leftrightarrow y$ and assume generically that $m + u \neq 0$. We thus have
\begin{equation}
  f(x, y) = x^2 y + x y^2 - \frac{y^2}{m} - \frac{(2u + m) x y}{u m} + m x + \frac{(m + u)y}{m^2 u}.
\end{equation}
Let us homogenize by writing $x = X/Z$, $y = X/Z$ and clearing denominators. We get,
\begin{equation}
  F(X, Y, Z) = F_3(X, Y) + F_2(X, Y) Z + F_1(X, Y) Z^2.
\end{equation}
where
\begin{align}
  F_3 &= X^2 Y + X Y^2, \\
  F_2 &= -\frac{(2u + m)}{u m} X Y - \frac{Y^2}{m}, \\
  F_1 &= m X + \frac{(m+u)}{m^2 u}  Y.
\end{align}
Note that the point $P = [0, 0, 1]$ is rational by construction (after eliminating the constant term). The tangent at $P$ is given by $F_1 = m X + \frac{(m+u)}{m^2 u} Y = 0$ and meets the curve at another point $Q$. We also define $e_i = F_i\qty(\frac{(m+u)}{m^2 u}, -m)$ ($e_1 = 0$ by definition) ---
\begin{align}
  e_2 &= \frac{2}{m^2}+\frac{3}{m u}-m+\frac{1}{u^2}, \\
  e_3 &= \frac{(m+u) \left(u-\frac{m+u}{m^3}\right)}{u^2}.
\end{align}
The coordinates of $Q$ are given by
\begin{equation}
  \begin{split}
    Q = \Big[& \frac{(m+u) \left(m^3 u^2-m^2-3 m u-2 u^2\right)}{m^4 u^3}, \\
    &-m^2+\frac{m}{u^2}+\frac{2}{m}+\frac{3}{u},\quad \frac{(m+u) \left(u-\frac{m+u}{m^3}\right)}{u^2} \Big].
  \end{split}
\end{equation}
For the curve to be elliptic, $e_2$ and $e_3$ cannot vanish simultaneously which is true provided $m$ and $u$ don't vanish simultaneously. On the real section of the $m$ -- $u$ plane, these conditions are plotted in \cref{fig:e2e3Vanish}.

\begin{figure}[htbp]
  \centering
  % first subfigure
  \begin{subfigure}[t]{0.4\textwidth}
    \centering
    \includegraphics[width=\linewidth]{figures/e3Vanish}
    \caption{$e_3 = 0$}
    \label{fig:e3Vanish}
  \end{subfigure}
  \hfill
  % second subfigure
  \begin{subfigure}[t]{0.4\textwidth}
    \centering
    \includegraphics[width=\linewidth]{figures/e2Vanish}
    \caption{$e_2 = 0$}
    \label{fig:e2Vanish}
  \end{subfigure}

  \caption{The loci where $e_2$, $e_3$ vanish.}
  \label{fig:e2e3Vanish}
\end{figure}

Assuming generically that $e_3 \neq 0$, we make the coordinate change
\begin{align}
  X &= X' + \frac{m u \left(\frac{2}{m^2}+\frac{3}{m u}-m+\frac{1}{u^2}\right)}{m+u-m^3 u} Z' \\
  Y &= Y' - \frac{m^2 \left(m^3 \left(-u^2\right)+m^2+3 m u+2 u^2\right)}{(m+u) \left(m+u-m^3 u\right)} Z' \\
  Z &= Z'
\end{align}
which shifts $Q$ to the origin, $[X', Y', Z'] = [0, 0, 1]$. We now shift back to affine coordinates $x' = X'/Z'$, $y' = Y'/Z'$. The curve can now be written as
\begin{equation}
  f'(x', y') = f_3'(x', y') + f_2'(x', y') + f_1'(x', y')
\end{equation}
where
\begin{align}
  f_3' \ = \ & x'^2 y' + x' y'^2 \\
  f_2' \ = \ & \frac{(m+u)^2}{m u \left(m+u-m^3 u\right)} y'^2 + \qty(\frac{2 \left(m+u-m^3 u\right)}{m (m+u)}+\frac{1}{u}) x' y' \notag \\
  & + \frac{m^2 \left(-\left(m^3-2\right) u^2+m^2+3 m u\right)}{(m+u) \left(\left(m^3-1\right) u-m\right)} x'^2 \\
  f_1' \ = \ & \frac{(m+u) \left(2 m^6 u^2-m^5-5 m^4 u-4 m^3 u^2+m^2+2 m u+u^2\right)}{m^2 u \left(m+u-m^3 u\right)^2} y' \notag \\
  & + \frac{m}{u (m+u)^2 \left(m+u-m^3 u\right)^2} \bigl( -m^5-6 m^4 u+\left(m^3-15\right) m^3 u^2 \notag \\
  & \qquad -3 \left(m^6-3 m^3+4\right) m u^4+\left(6 m^3-19\right) m^2 u^3 \notag \\
  & \qquad +\left(m^9-3 m^6+4 m^3-3\right) u^5 \bigr) x'
\end{align}
We define $\phi_i = f_i'(1, t)$ and
\begin{equation}
  \begin{split}
    \delta &= \phi_2^2 - 4 \phi_1 \phi_3 \\
    &= \frac{(m+u)^4}{m^2 u^2 \left(m+u-m^3 u\right)^2} t^4 \\
    & \quad - \frac{2 (m+u) \left(-\left(5 m^3+1\right) u^2-m \left(m^3+2\right) u-m^2+2 m^2 \left(m^3-2\right) u^3\right)}{m u^2 \left(m+u-m^3 u\right)^2} t^3 \\
    & \quad + \frac{1}{u^2 \left(m+u-m^3 u\right)^2} \bigl( 2 m \left(2 m^3+1\right) u+m^2+2 m \left(-2 m^6+m^3+4\right) u^4 \\
    & \qquad +\left(m^6+18 m^3+1\right) u^2+2 m^2 \left(m^3+11\right) u^3 \bigr) t^2 \\
    & \quad + \frac{2 m}{u (m+u) \left(m+u-m^3 u\right)^2} \bigl( m^4+\left(m^3+4\right) m^3 u+2 \left(-m^6+m^3+1\right) u^4 \\
    & \qquad +\left(-3 m^6+7 m^3+6\right) m u^3+\left(6 m^3+7\right) m^2 u^2 \bigr) t \\
    & \quad + \frac{m^4 \left(-\left(m^3-2\right) u^2+m^2+3 m u\right)^2}{(m+u)^2 \left(m+u-m^3 u\right)^2}
  \end{split}
\end{equation}
Note that $\delta = 0$ whenever $t$ is the slope of a tangent to the curve that passes through $Q$, in particular for $t = -\frac{m^3 u}{m+u}$ (can be checked). We write $t = \frac{1}{\tau} -\frac{m^3 u}{m+u}$ and define $\rho = \tau^4 \delta$ which is a cubic in $\tau$
\begin{equation}
  \begin{split}
    \rho \ = \ &\frac{4 m \left(m+u-m^3 u\right)}{m+u} \tau^3 + \qty(8 m+\frac{1}{u^2}) \tau^2 + \frac{2 (m+u) \left(2 m^2 u^2+m+u\right)}{m u^2 \left(m+u-m^3 u\right)} \tau \\
    & + \frac{(m+u)^4}{m^2 u^2 \left(m+u-m^3 u\right)^2}.
  \end{split}
\end{equation}
Applying the final transformation,
\begin{align}
  \rho &= \frac{(m+u)^2}{m^2 \left(m+u-m^3 u\right)^2} V^2, \\
  \tau &= \frac{(m+u)}{m \left(m+u-m^3 u\right)} \bigl( U - \frac{1}{12} \qty(8 m+\frac{1}{u^2}) \bigr),
\end{align}
we obtain the final Weierstrass form,
\begin{equation}
  V^2 = 4 U^3 - g_2 U + g_3,
\end{equation}
with
\begin{align}
  g_2 &= -\frac{4}{3} \qty(-16 m^2+\frac{8 m}{u^2}-\frac{1}{u^4}+\frac{24}{u}), \\
  g_3 &= -\frac{8}{27} \qty(-64 m^3+\frac{48 m^2}{u^2}-\frac{12 m}{u^4}+\frac{144 m}{u}+\frac{1}{u^6}-\frac{36}{u^3}+216).
\end{align}


\subsection{Rational change of coordinates}

We saw in the last section that Nagell's algorithm turns the local mirror of $\mathbb{F}_1$ into Weierstrass form but gives a very complicated transcendental change of coordinates. A simpler change of coordinates is given by
\begin{equation}
  
\end{equation}


\section{Testing N\'eron \texorpdfstring{$N$}{N}-gon result for \texorpdfstring{$\mathbb{F}_0$}{F0}}

We write the mirror curve to $\mathbb{F}_0$ as the following embedding in $\mathbb{C}^* \times \mathbb{C}^* \ni (w, X)$
\begin{equation}
  w + \frac{1}{w} + R^2 \qty(X + \frac{1}{X} + U) = 0.
\end{equation}
We can write this curve as a quartic by introducing $Y = -R^2 X \qty(w - \frac{1}{w})$,
\begin{equation}\label{eqn:F0quartic}
  Y^2 = f_4(X) = (X^2 + U X + 1)^2 - 4 X^2 R^2.
\end{equation}
The discriminant $\delta$ turns out to be (upto a numerical constant)
\begin{equation}
  \delta = R^8 \, \Delta(U, R), \quad \Delta(U, R) = 16 R^8-8 R^4 \left(U^2+4\right)+\left(U^2-4\right)^2.
\end{equation}
The equation $\Delta(U, R) = 0$ has four solutions
\begin{equation} \label{eqn:delt0sol}
  U_1 = 2 - 2R^2, \quad U_2 = -2 + 2R^2, \quad U_3 = -2 - 2R^2, \quad U_4 = 2 + 2 R^2.
\end{equation} 
In the real $U-R$ slice, the condition is plotted in \cref{fig:Delt0}. 
\begin{figure}[ht]
  \centering
  \includegraphics[width=.7\textwidth]{Delt0}
  \caption{The locus $\Delta(U, R) = 0$}
  \label{fig:Delt0}
\end{figure}

\subsection{Cubic Form}
We can turn the quartic in \cref{eqn:F0quartic} to cubic form by the following rational transformation
\begin{equation}
  X = \frac{a x + b}{c x + d}, \quad Y = \frac{a d - b c}{(c x + d)^2} y.
\end{equation}
The curve now looks like
\begin{equation}
  Y^2 = \frac{(a d - b c)^2}{(c x + d)^4} y^2 = f_4\qty(\frac{a x + b}{c x + d}).
\end{equation}
This implies,
\begin{equation}
  (a d - b c)^2 y^2 = f_4\qty(\frac{a x + b}{c x + d})(c x + d)^4 = c^4 f_4\qty(\frac{a}{c}) x^4 + f_3(x),
\end{equation}
where $f_3$ is a cubic polynomial in $x$ whose coefficient of $x^3$ is directly proportional to $f_4'(a/c)$. It follows that if we set $a/c$ to be a simple root of $f_4$ and $a d - b c = 1$ then we obtain the desired cubic form. We can then scale and translate so as to get the cubic in the usual Weierstrass form: $y^2 = 4 x^3 - g_2 x - g_3$.

Performing this procedure on the quartic in \cref{eqn:F0quartic}, one gets
\begin{equation}
  \begin{aligned}
    g_2 &= \frac{1}{12} \left(16 R^8-8 R^4 \left(U^2+2\right)+\left(U^2-4\right)^2\right), \\
    g_3 &= \frac{1}{216} \left(4 R^4-U^2+4\right) \left(16 R^8-8 R^4 \left(U^2+5\right)+\left(U^2-4\right)^2\right).
  \end{aligned}
\end{equation}
We also find the $j$-invariant defined as $j = \frac{g_2^3}{g_2^3 - g_3^2}$,
\begin{equation}
  j = \frac{\left(16 R^8-8 R^4 \left(U^2+2\right)+\left(U^2-4\right)^2\right)^3}{R^8 \left(16 R^8-8 R^4 \left(U^2+4\right)+\left(U^2-4\right)^2\right)}.
\end{equation}


\subsection{Nodal curve}

As shown in \cref{eqn:delt0sol}, the curve in \cref{eqn:F0quartic} develops four distinct branches of nodes. These correspond to points in parameter space where the elliptic curve degenerates. Following the procedure outlined in \cite{kerr2008}, we rationally parametrize each nodal curve using a parameter $z$, such that the limits $z \to 0$ and $z \to \infty$ map to the node of the curve.

The quartic polynomial in \cref{eqn:F0quartic} takes the following forms along the four branches, each associated with a corresponding node:
\begin{equation}
  \begin{aligned}
    U_1 &= 2 - 2R^2, &\qquad f_4(X) = (X+1)^2 \left(X^2 + \qty(2-4 R^2) X+1\right), && X_0 = -1, \\
    U_2 &= -2 + 2R^2, &\qquad f_4(X) = (X-1)^2 \left(X^2 + \qty(4 R^2-2) X+1\right), && X_0 = 1, \\
    U_3 &= -2 - 2R^2, &\qquad f_4(X) = (X-1)^2 \left(X^2+ \qty(-4 R^2-2) X+1\right), && X_0 = 1, \\
    U_4 &= 2 + 2 R^2, &\qquad f_4(X) = (X+1)^2 \left(X^2+\qty(4 R^2+2) X+1\right), && X_0 = -1.
  \end{aligned}
\end{equation}

To obtain a rational parametrization for the first branch ($U_1$), it suffices to find a rational expression for $X$ such that $\left(X^2 + \qty(2-4 R^2) X+1\right)$ becomes a perfect square. One convenient choice is
\begin{equation}
  X = \frac{2 \left(-2 R^2-t+1\right)}{t^2-1},
  \qquad
  \Rightarrow \qquad
  \left(X^2 + \qty(2-4 R^2) X+1\right) = 
  \qty(\frac{\left(4 R^2-2\right) t+t^2+1}{t^2-1})^2.
\end{equation}
The remaining parametrizations follow from the transformations $R^2 \to -R^2$ and $X \to -X$:
\begin{equation}
\begin{aligned}
  X_1 &= \frac{2 \left(-2 R^2-t+1\right)}{t^2-1}, &\quad
  Y_1 = \frac{-16 R^4 t+4 R^2 (t-1)^3+(t-1)^4}{\left(t^2-1\right)^2}, \\
  X_2 &= -\frac{2 \left(-2 R^2-t+1\right)}{t^2-1}, &\quad
  Y_2 = \frac{-16 R^4 t+4 R^2 (t-1)^3+(t-1)^4}{\left(t^2-1\right)^2}, \\
  X_3 &= -\frac{2 \left(2 R^2-t+1\right)}{t^2-1}, &\quad
  Y_3 = \frac{-16 R^4 t-4 R^2 (t-1)^3+(t-1)^4}{\left(t^2-1\right)^2}, \\
  X_4 &= \frac{2 \left(2 R^2-t+1\right)}{t^2-1}, &\quad
  Y_4 = \frac{-16 R^4 t-4 R^2 (t-1)^3+(t-1)^4}{\left(t^2-1\right)^2}.
\end{aligned}
\end{equation}
Each pair $(X_i, Y_i)$ satisfies \cref{eqn:F0quartic} for the corresponding value of $U_i$.

Returning to the original toric coordinates $(X, w)$, the parametrizations become
\begin{equation}
  \begin{aligned}
  w_1 &= \frac{(t-1) \left(2 R^2+t-1\right)}{2 R^2 (t+1)}, \quad
  w_2 = -\frac{(t-1) \left(2 R^2+t-1\right)}{2 R^2 (t+1)}, \\
  w_3 &= \frac{(t-1) \left(2 R^2-t+1\right)}{2 R^2 (t+1)}, \quad
  w_4 = -\frac{(t-1) \left(2 R^2-t+1\right)}{2 R^2 (t+1)}.
  \end{aligned}
\end{equation}

This yields a rational parametrization of the nodal curve in toric coordinates, though the node still lies at a finite distance rather than at $0$ or $\infty$. To remedy this, we perform one final change of variables. For $U_1$, the node at $X_0 = -1$ corresponds to $t = 1 \pm 2R$, so we define
\begin{equation}
  z = \frac{t - (1-2R)}{t - (1+2R)},
\end{equation}
similarly, for $U_2$. For $U_3$ and $U_4$, the node corresponds to $t = 1 \pm 2 \ii R$, so we define
\begin{equation}
  z = \frac{t - (1-2\ii R)}{t - (1+2\ii R)}.
\end{equation}
Therefore, for each branch we get $X$ and $w$ in the $z$-parametrization:
\begin{equation}\label{eqn:F0rationalparam}
  \begin{aligned}
    X_1 &= -\frac{(z-1) (R (z-1)+z+1)}{(z+1) (R z+R+z-1)}, &\quad
    w_1 = \frac{(z+1) (R (z-1)+z+1)}{(z-1) (R z+R+z-1)}, \\
    X_2 &= \frac{(z-1) (R (z-1)+z+1)}{(z+1) (R z+R+z-1)}, &\quad
    w_2 = -\frac{(z+1) (R (z-1)+z+1)}{(z-1) (R z+R+z-1)}, \\
    X_3 &= \frac{(z-1) (R (z-1)+i (z+1))}{(z+1) (R (z+1)+i (z-1))}, &\quad
    w_3 = \frac{(z+1) (R (z-1)+i (z+1))}{(z-1) (R (z+1)+i (z-1))}, \\
    X_4 &= -\frac{(z-1) (R (z-1)+i (z+1))}{(z+1) (R (z+1)+i (z-1))}, &\quad
    w_4 = -\frac{(z+1) (R (z-1)+i (z+1))}{(z-1) (R (z+1)+i (z-1))}.
  \end{aligned}
\end{equation}

Applying the method of \cite[Section~4.2]{kerr2008}, we find for the first branch the exponents and roots of the linear factors (with $i=1$, $j = 1,\dots,4$):
\begin{equation}
  \begin{aligned}
    (d_j) &= (1, 1, -1, -1), &\quad (\alpha_j) = \qty(1,\frac{R-1}{R+1},-1,\frac{1-R}{R+1}), \\
    (e_j) &= (1, 1, -1, -1), &\quad (\beta_j) = \qty(-1,\frac{R-1}{R+1},1,\frac{1-R}{R+1}).
  \end{aligned}
\end{equation}

The (imaginary part of the) non-vanishing period at the conifold point is then
\begin{equation}
  \mathcal{P}(R) = \frac{1}{2\pi} \sum_{j, k} d_j e_k D_2\qty(\frac{\alpha_j}{\beta_k}),
\end{equation}
where $D_2(z)$ is the Bloch--Wigner function,
\begin{equation}
  D_2(z) = \Im\!\qty(\operatorname{Li}_2(z)) + \arg(1-z)\,\log\abs{z},
\end{equation}
and $\operatorname{Li}_2(z)$ denotes the ordinary dilogarithm \cite{zagier:dilogarithm}.

We thus find, up to an overall sign,
\begin{equation}
  \mathcal{P}(R) = \frac{1}{\pi} \qty[-D_2\!\left(\frac{R+1}{1-R}\right)
  + D_2\!\left(\frac{R+1}{R-1}\right)
  + D_2\!\left(\frac{1-R}{R+1}\right)
  - D_2\!\left(\frac{R-1}{R+1}\right)].
\end{equation}
Using the Bloch--Wigner identities \cite[Eq.~(3)]{zagier:dilogarithm},
\begin{equation}
  D_2(z) = D_2\!\qty(1-\frac{1}{z}) = D_2\!\qty(\frac{1}{1-z})
  = -D_2\!\qty(\frac{1}{z}) = -D_2(1-z)
  = -D_2\!\qty(\frac{-z}{1-z}),
\end{equation}
we note that
\begin{equation}
  D_2\!\qty(\frac{1-R}{1+R}) = -D_2\!\left(\frac{R+1}{1-R}\right),
  \qquad
  D_2\!\left(\frac{R-1}{R+1}\right) = -D_2\!\left(\frac{R+1}{R-1}\right),
\end{equation}
so that
\begin{equation}\label{eqn:realPeriod}
  \mathcal{P}(R) = \frac{2}{\pi}
  \qty[D_2\!\left(\frac{R+1}{R-1}\right)
  - D_2\!\left(\frac{R+1}{1-R}\right)].
\end{equation}
Since $D_2(\bar{z}) = -D_2(z)$, it follows that
\begin{equation}
  \mathcal{P}(R) = 0 \qquad \text{for real } R.
\end{equation}

Repeating the analysis for $U_2$ yields identical exponents and roots, and thus the same result. For $U_3$ and $U_4$, the change of variables is $t = \frac{-2 \ii R (z+1)+z-1}{z-1}$, giving
\begin{equation}
  \begin{aligned}
    (d_j) &= (1, 1, -1, -1), &\quad
    (\alpha_j) = \qty(1,\frac{R-\ii}{R+\ii},-1,\frac{\ii-R}{R+\ii}), \\
    (e_j) &= (1, 1, -1, -1), &\quad
    (\beta_j) = \qty(-1,\frac{R-\ii}{R+\ii},1,\frac{\ii-R}{R+\ii}).
  \end{aligned}
\end{equation}
Hence,
\begin{equation}\label{eqn:imagPeriod}
  \begin{aligned}
    \mathcal{P}(R)
    &= \frac{1}{\pi}\qty[-D_2\!\left(\frac{R+i}{-R+i}\right)
      + D_2\!\left(\frac{R+i}{R-i}\right)
      + D_2\!\left(\frac{-R+i}{R+i}\right)
      - D_2\!\left(\frac{R-i}{R+i}\right)] \\
    &= \frac{2}{\pi} \qty[D_2\!\left(\frac{-R+i}{R+i}\right)
      - D_2\!\left(\frac{R-i}{R+i}\right)].
  \end{aligned}
\end{equation}
And this time we see that
\begin{equation}
  \mathcal{P}(R) = 0 \qquad \text{for imaginary } R.
\end{equation}

We want to now understand how (and/or if) the formulas \cref{eqn:realPeriod,eqn:imagPeriod} reproduce the formulas for the 'quantum volumes' in \cite[eq. (A.85) and eq. (A.107)]{LeFloch:2024cwl} which we copy below for reference after identifying $\lambda = e^{2\pi\ii m}$ and from the parametrization of the mirror curve, we identify $R = \lambda^\frac{1}{4} = e^{\ii \pi m/2}$,
\begin{align}
  \mathcal{V}(R) &= \frac{2 \ii}{\pi^2} \qty(\Li_2(-\ii R) - \Li_2(\ii R)), \\
  \tilde{\mathcal{V}}(R) &= \frac{2 \ii}{\pi^2} \qty(\Li_2(-R) - \Li_2(R)) = \mathcal{V}(-\ii R).
\end{align}

Since $D_2$ is the imaginary part of $\Li_2$, we must look at the full non-vanishing period to make this identification. 

\subsection{Full period}
In fact, apart from the imaginary part of the period, \cite[eq. (4.12)]{kerr2008} also gives a formula for the full non-vanishing period at the conifold points. We copy it here
\begin{equation}\label{eqn:fullperiodkerrdonan}
\begin{aligned}
  \mathcal{Q} &= \frac{1}{2\pi} \sum_{j, k} d_j e_k \qty[ \Li_2\qty(\frac{\alpha_j}{\beta_k}) + \qty(\log \alpha_j - \log \beta_k) \log(1 - \frac{\alpha_j}{\beta_k}) ] \\
  &\quad -\frac{1}{2\pi} \log y(0) \qty(\log x(0) - \log x(\infty))
  + \log y(0) \sum_j d_j \log \alpha_j \\
  &\quad - \log x(\infty) \sum_k e_k \log \beta_k.
\end{aligned}
\end{equation}
where as before, $x(z)$ and $y(z)$ are rational parametrizations of the nodal curve in the original toric coordinates and the node corresponds to $z \to 0, \infty$.
From now on, we work with the branches $U_1$ and $U_2$ but all the results can be generalized to $U_3$ and $U_4$ by taking $R \to \ii R$.
Evaluating \cref{eqn:fullperiodkerrdonan} for the parametrization in \cref{eqn:F0rationalparam}, we find the following after assuming $\Im R > 0$ and simplifying logs
\begin{equation}
  \begin{aligned}
    \mathcal{Q} &= \frac{1}{\pi} \qty(\Li_2\left(\frac{-R-1}{R-1}\right)-\Li_2\left(\frac{R+1}{R-1}\right)-\Li_2\left(\frac{1-R}{R+1}\right)+\Li_2\left(\frac{R-1}{R+1}\right)) \\
    &\quad -\frac{2}{\pi} \log (R) \log (R-1)+i  \log (R-1)+2 i  \log (R)\\
    &\quad +\frac{2}{\pi} \log (R) \log (R+1)-i  \log (R+1).
    \end{aligned}
\end{equation}

This expression can be simplified further thanks to various identities of the dilogarithm. We use in particular the following identities \cite{zagier:dilogarithm}
\begin{align}
  \Li_2\qty(\frac{1}{z}) &= -\Li_2(z) - \frac{\pi^2}{6} - \frac{1}{2}\log(-z)^2, \label{eqn:Li2Inversion}\\
  \Li_2\qty(1-z) &= -\Li_2(z) + \frac{\pi^2}{6} - \log(z) \log(1-z). \label{eqn:Li2Reflection}
\end{align}
We also have the so-called \textit{five term relation},
\begin{equation}
  \begin{aligned}
    &\Li_2(x)+\Li_2(y)+\Li_2\left(\frac{x-1}{x y-1}\right)+\Li_2\left(\frac{y-1}{x y-1}\right)+\Li_2(1-x y) \\
    &= \frac{\pi ^2}{2} - \log (1-x) \log (x)- \log (1-y) \log (y) -\log \left(\frac{1-x}{1-x y}\right) \log \left(\frac{1-y}{1-x y}\right),
  \end{aligned}
\end{equation}
which only holds near the viscinity of $x, y \in (0, 1)$ due to branch cuts. However, it is easy to find a generalized version of this identify for the domain we are interested in by starting with the identity in it's twice differentiated form (where it reduces to a trivial algebraic identity) and integrating twice and fixing the integration constants.

Using the inversion identity \cref{eqn:Li2Inversion}, we get
\begin{equation}
  \begin{aligned}
  \mathcal{Q} &= -\frac{\pi}{2} -\frac{2}{\pi} \qty[\Li_2\left(\frac{1-R}{R+1}\right)- \Li_2\left(\frac{R-1}{R+1}\right) -  \log \left(\frac{1-R}{R+1}\right) \log (R)].
  \end{aligned}
\end{equation}
We now invoke the five term relation with $(x, y) = (R, -1)$ and correct it with the procedure explained. After applying the reflection identity to some of the arguments \cref{eqn:Li2Reflection} and noting that $\Li_2(-1) = \frac{-\pi^2}{12}$, the identity reads simply
\begin{equation}
  -\Li_2(-R)+\Li_2(R)+\Li_2\left(\frac{1-R}{R+1}\right)-\Li_2\left(\frac{R-1}{R+1}\right)
  = \frac{\pi ^2}{4}-\log (R) \log \left(\frac{1-R}{R+1}\right).
\end{equation}
Applying this to $Q$, we get our final expression
\begin{equation}
  \mathcal{Q} = - \pi - \frac{2}{\pi} \qty(\Li_2(-R)-2 \Li_2(R)).
\end{equation}


\printbibliography[title={References}]

\end{document}